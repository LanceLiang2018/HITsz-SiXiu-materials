\nwtitle{《思想道德修养与法律基础》课程复习纲要}
\begin{center}
本文档由Hans Wan整理并开源至GitHub:\verb|https://github.com/criwits/HITsz-SiXiu-materials|
\end{center}

\section{人生的青春之问}

\subsection{人生观的主要内容及其之间的关系}
\begin{enumerate}
\item 人生观的主要内容包括人生目的、人生态度和人生价值。
\item 人生目的是指生活在一定历史条件下的人在人生实践中关于自身行为的根本指向和人生追求,是对“人为什么活着”这一人生根本问题的认识和回答,是人生观的核心。
\item 人生态度是指人们通过生活实践形成的对人生问题的一种稳定的心理倾向和精神状态,是人生观的重要内容。
\item 人生价值是指人的生命及其实践活动对于社会和个人所具有的作用和意义。
\item 总之,人生目的表明人的一生追求什么,人生态度表示以怎样的心态实现人生目标,人生价值判定一个具体人生的价值和意义。人生目的决定人们对待实际生活的基本态度和人生价值的评判标准,人生态度影响着人们对人生目的的持守和人生价值的评判,人生价值制约着人生目的和人生态度的选择。
\end{enumerate}

\subsection{人生价值的两个方面以及评价人生价值的科学方法}
\begin{enumerate}
\item 人生价值的两个方面是:
\begin{itemize}
\item 人生的自我价值,即个体的人生活动对自己的生存和发展所具有的价值,主要表现为对自身物质和精神需要的满足程度。
\item 人生的社会价值,即个体的实践活动对社会、他人所具有的价值。
\item 人生的自我价值和社会价值,既相互区别,又密切联系、相互依存,共同构成人生价值的矛盾统一体。
\end{itemize}
\item 评价人生价值的科学方法有:
\begin{itemize}
\item 坚持能力有大小与贡献须尽力相统一。考察一个人的人生价值,要把个人对社会的贡献同他的能力以及与能力相对应的职责联系起来。
\item 坚持物质贡献与精神贡献相统一。评价人生价值,既要看一个人对社会作出的物质贡献,也要看他对社会作出的精神贡献。
\item 坚持完善自身与贡献社会相统一。人生的社会价值是实现人生自我价值的基础。评价人生价值的大小应主要看一个人对社会所作的贡献,但这并不意味着要否定人生的自我价值。
\end{itemize}
\end{enumerate}

\subsection{辩证对待人生矛盾}
人生中矛盾是无处不在的。我们需要科学认识实际生活中的各种问题,勇敢面对和正确处理各种人生矛盾。
\begin{enumerate}
\item 树立正确的幸福观。我们应该思考,什么是人生的真正幸福,我们应该追求什么样的幸福,通过什么样的方式实现幸福。我们也需要清楚地认识到,实现幸福离不开一定的物质条件,更不能把自己的幸福建立在损害社会整体利益和他人利益的基础之上。只有在为社会做贡献、为他人服务的过程中,我们才能产生更大的幸福感。
\item 树立正确的得失观。我们要以积极进取的态度去面对生活中的成败得失,使一时的失败成为人生的财富而不是人生的包袱。我们不要拘泥于个人利益的得失,不要满足于一时的“得”,也不要惧怕一时的“失”。
\item 树立正确的苦乐观。苦与乐既对立又统一,在一定条件下可以相互转化。我们一块准确把我苦与乐的辩证关系,努力做迎难而上、艰苦奋斗的开拓者。
\item 树立正确的顺逆观。我们要善于利用顺境,勇于正视逆境和战胜逆境,才能够实现自己的人生价值。
\item 树立正确的生死观。生与死是贯穿人生始终的一对基本矛盾,我们应该珍惜韶华,在服务人民、投身民族复兴的伟大事业中开发出生命所蕴含的巨大潜能,努力给有限的个体生命赋予更有价值的意义。
\item 树立正确的荣辱观。我们应该坚持“八荣八耻”,明确在纷繁复杂的社会生活中应当坚持和提倡什么、反对和抵制什么,从而为自身判断行为得失,做出道德选择,确定价值取向提供基本的价值准则和行为规范。
\end{enumerate}

\section{坚定理想信念}

\subsection{理想信念是“精神之钙”}
理想指引方向,信念决定成败。理想信念是人生发展的内在动力。我们作为大学生,在提高知识水平。增强实践才干的同时,更要坚定崇高的理想信念。
\begin{enumerate}
\item 理想信念昭示奋斗目标。我们只有树立起崇高的理想信念,才能够解答好人生的意义、奋斗的价值以及做什么样的人等重要的人生课题。
\item 理想信念提供前进动力。大学时期确立起理想信念,对今后的人生之路将产生重要影响,甚至会影响终身。例如,大学生人生目标的确立、生活态度的形成、知识才能的丰富、发展方向的设定、工作岗位的选择,以及如何择友、如何面对挫折、如何克服困难等问题的解决,都离不开坚定的理想信念。
\item 理想信念提高精神境界。在追求理想和实现理想的过程当中,我们会不断面对各种挑战、抵御各种诱惑、突破各种局限、克服各种困难,而这个过程也是我们的精神境界从狭隘走向高远、从空虚走向充实、从犹疑走向执着的过程。
\end{enumerate}

大学生只有树立崇高的理想信念,才能激发起为民族复兴和人民幸福而发奋学习的强烈责任感和使命感。

\subsection{为什么要信仰马克思主义}
马克思主义作为我们立党立国的根本指导思想,是近代以来中国发展的必然结果,是中国人民长期探索的历史选择,也是由马克思主义严密的科学体系、鲜明的阶级立场和巨大的实践指导作用决定的。
\begin{enumerate}
\item 马克思主义体现了科学性和革命性的统一。马克思主义深刻地揭示了自然界、人类社会和人类思维发展的普遍规律,揭示了事物的本质、内在联系和发展规律,是“伟大的认识工具”,是人们观察世界、分析问题的有力思想武器。
\item 马克思主义具有鲜明的实践品格。马克思主义不仅致力于科学地解释世界,而且致力于积极地改变世界。在人类思想史上,还没有一种理论像马克思主义那样对人类文明进步产生如此广泛而巨大的影响。
\item 马克思主义具有持久生命力。作为一个开放的理论体系,马克思主义不断地吸收、提炼人类创造的一切科学知识和文明成果,并将其运用于推进历史的进步。无论时代如何变迁、科学如何进步,马克思主义依然占据着真理和道义的制高点,仍然具有强大持久的生命活力。
\end{enumerate}

大学生只有树立马克思主义的科学信仰,才能真正确立崇高的理想信念,在错综复杂的社会现象中看清本质、明确方向,为服务人民、奉献社会作出更大的贡献。

\subsection{中国特色社会主义是我们的“共同理想”,它与我们的“远大理想”之间有怎样的关系}
\begin{enumerate}
\item 实现中国特色社会主义是我们的共同理想,实现共产主义是我们的远大理想。
\item 我们要牢固确立在中国共产党领导下走中国特色社会主义道路、为实现中华民族伟大复兴而奋斗的共同理想和坚定信念。
\item 而共产主义远大理想的实现是一个漫长、艰辛的历史过程,也需要我们一代又一代人的不懈奋斗和接续努力。
\item 实现共产主义是我们的远大理想,坚持和发展中国特色社会主义,就是向着远大理想所进行的实实在在的努力。
\end{enumerate}

走好新时代的长征路,大学生要不断增强中国特色社会主义道路自信、理论自信、制度自信、文化自信,自觉做共产主义远大理想和中国特色社会主义共同理想的坚定信仰者、忠实实践者,为崇高理想信念而矢志奋斗。

\subsection{个人理想与社会理想的统一}
个人理想指处于一定历史条件和社会关系中的个体对于自己未来的物质生活、精神生活所产生的种种向往和追求。社会思想指社会集体乃至社会全体成员的共同思想,即在全社会占主导地位的共同奋斗目标。社会理想与个人理想不是彼此孤立的,而是相互联系、相互影响和相互制约的。
\begin{enumerate}
\item 个人理想以社会理想为指引。个人理想的确立要以社会理想为主导,个人理想的实现依赖于社会理想的实现。我们的个人理想只有同国家的前途、民族的命运相结合,个人的向往和追求只有同社会的需要和人民的利益相一致,才可能变为现实。
\item 社会理想是对个人理想的凝练和升华。社会理想不是凭空产生的,也不是由外在力量强加的,而是建立在众人的个人理想基础之上。
\end{enumerate}

我们作为大学生,要在社会思想的指引下,珍惜韶华、奋发有为,勇于追求个人理想,在实现社会理想的过程中努力实现个人理想。

\section{弘扬中国精神}
\subsection{民族精神的内涵(“四个伟大”)}
民族精神是一个民族在长期共同生活和社会实践中形成的,为本民族大多数成员所认同的价值取向、思维方式、道德规范、精神气质的总和,是一个民族赖以生存和发展的精神支柱。
\begin{enumerate}
\item 伟大创造精神。中国人民自古以来始终辛勤劳作、发明创造。我国产生了老子、孔子、庄子、孟子等伟大思想巨匠,发明了造纸术、火药、印刷术等伟大科技成果,创作了《诗经》、楚辞、汉赋等伟大文艺作品,建设了万里长城、布达拉宫、故宫等伟大工程。而今天,中国人民的伟大创造精神正在前所未有地迸发出来,推动我国经济日新月异地向前发展。
\item 伟大奋斗精神。在几千年历史长河中,中国人民始终革故鼎新、自强不息。中国人民自古就明白,要幸福就要奋斗;而今天,只要13亿多中国人民始终发扬这种伟大奋斗精神,就一定能够达到创造人民更加美好生活的宏伟目标。
\item 伟大团结精神。自古以来,中国人民始终团结一心、同舟共济。今天,中国取得的令世人瞩目的发展成就,更是全国各族人民同心同德、同向努力的结果。
\item 伟大梦想精神。中国人民自古以来始终心怀梦想、不懈追求。中国人民相信,山再高,往上攀,总能登顶;路再长,走下去,定能到达。这种伟大的梦想精神,是支持我们不断奋斗,奋勇向前的精神力量。
\end{enumerate}

勤劳勇敢的中国人民培育、继承、发展起来的以爱国主义为核心的伟大民族精神,是坚定中国特色社会主义道路自信、理论自信、制度自信、文化自信地底气,是中华民族风雨无阻、高歌行进的根本力量。

\subsection{爱国主义的基本内涵及其时代要求}
\begin{enumerate}
\item 爱国主义体现了人们对自己祖国的深厚感情,揭示了个人对祖国的依存关系,是人们对自己家园以及民族和文化的归属感、认同感、尊严感和荣誉感的统一。它的主要内涵有:
\begin{itemize}
\item 爱祖国的大好河山。祖国的河山在人们的心中占据着至高无上的地位,它不只是自然风光,更是主权、财富、民族发展和进步的基本载体。每一个爱国者都应该把“保我国土”“爱我家乡”、维护祖国领土的完整和统一作为自己的神圣使命和义不容辞的责任。
\item 爱自己的骨肉同胞。中华民族的利益是我国各族人民的共同利益、长远利益和最高利益,这种利益高于各个民族内部的、局部的、暂时的利益。我们要始终坚持以人民为中心的立场,始终紧紧地同人民群众站在一起。
\item 爱祖国的灿烂文化。文化是一个国家、一个民族的灵魂。我们要在充分理解和尊重的基础之上,积极推动祖国优良历史文化传统的传承和发展。
\item 爱自己的国家。我们要拥护国家的基本制度,遵守国家的宪法法律,维护国家安全和统一,捍卫国家的利益,为国家繁荣发展贡献自己的力量,这些都是爱国主义的基本要求。
\end{itemize}
\item 新时代的爱国主义,既承接了中华民族的爱国主义优良传统,又体现了鲜明的时代特征,内涵更加丰富。新时代的爱国主义要求:
\begin{itemize}
\item 坚持爱国主义和社会主义相统一。在当代中国,爱国主义首先体现在对社会主义的热爱上。社会主义制度的建立,为中国的繁荣发展提供了可靠的保障。爱国主义与爱社会主义的统一是中国历史发展的必然结果。而坚定拥护中国共产党的领导,是中华民族走向复兴、中国特色社会主义事业走向成功的必然要求,也是新时代爱国主义的必然要求。
\item 维护祖国统一和民族团结。维护和推进祖国统一,是中华民族走向伟大复兴的题中之义;而中华民族和各民族的关系,是一个大家庭和家庭成员的关系。只有维护好祖国统一和民族团结,我们才能弘扬新时代的爱国精神。
\item 尊重和传承中华民族历史和文化。对祖国悠久历史、深厚文化的理解和接受,是人们爱国主义情感培育和发展的重要条件。我们必须尊重和传承中华民族历史和文化,以时代精神激活中华民族优秀传统文化的生命力,增强文化自信,才能增强自己作为中国人的骨气和底气。
\item 必须坚持立足民族又面向世界。中国的命运与世界的命运紧密相关,弘扬新时代的爱国主义,必须既坚持立足民族,维护国家发展的主体性,又必须面向世界,构建人类命运共同体。
\end{itemize}
\end{enumerate}

\subsection{改革创新是时代要求}
改革创新是社会发展的重要动力,坚持改革创新是新时代的迫切要求。
\begin{enumerate}
\item 创新始终是推动人类社会发展的第一动力。从某种意义上说,创新决定着世界政治经济力量对比的变化,也决定着各国各民族的前途命运。
\item 创新能力是当今国际竞争新优势的集中体现。今天,国际竞争的新优势越来越集中体现在创新能力上,创新战略竞争在综合国力竞争中的地位日益重要。
\item 改革创新是我国赢得未来的必然要求。在新一轮科技革命和产业变革中,我国能否在未来发展中后来居上、弯道超车,主要就看能否在创新驱动发展上迈出实实在在的步伐。
\end{enumerate}

“聪者听于无声,明者见于未形。”大学生要自觉树立敢为天下先的志向和信心,敢于担当、勇于超越,在攻坚克难中追求卓越,在改革创新中引领世界潮流。

\subsection{做改革创新“生力军”}
新时代的大学生应当以时代使命为己任,把握时代的脉搏,迎接时代的挑战,增强创新创造的能力和本领,勇做改革创新的实践者,将弘扬改革创新精神贯穿于实践中、体现在行动上。
\begin{enumerate}
\item 树立改革创新的自觉意识。
\begin{itemize}
\item 增强改革创新的责任感。我们要不断增强以改革创新推动社会进步,在改革创新中奉献服务社会、实现人生价值的崇高责任感和使命感,以时不我待、只争朝夕的紧迫感投身改革创新的实践中。
\item 树立大胆探索未知领域的信心。我们应是常为新、敢创造的,理当锐意创新创造,不等待、不观望、不懈怠,勇做改革创新的生力军。
\end{itemize}
\item 增强改革创新的能力本领。
\begin{itemize}
\item 夯实创新基础。我们应从扎实系统的专业知识学习起步入手,而不能好高骛远,空谈改革,坐论创新。
\item 培养创新思维。我们在专业学习与社会实践中应自觉培养创新型思维,勤于思考,善于发现,勇于创新。
\item 投身创新实践。我们应当在全面深化改革的伟大实践中深深体悟改革创新精神,增强改革创新的意识,锤炼改革创新的意志,增强改革创新的能力本领,勇做改革创新的实践者和生力军。
\end{itemize}
\end{enumerate}

大学生应当珍惜人生中最具有创新创造力的宝贵时期,有敢为人先、开拓进取的锐气,有逢山开路、遇河架桥的意志,在创新创造中不断积累经验、取得成果、演绎精彩。

\section{践行社会主义核心价值观}
\subsection{社会主义核心价值观的基本内容}
\subsection{坚定核心价值观自信的原因}
\subsection{大学生如何做社会主义核心价值观的积极践行者}

\section{明大德、守公德、严私德}
\subsection{马克思主义道德的起源与本质}
\subsection{中国革命道德的主要内容和当代价值}
\subsection{社会主义道德建设的核心和原则}
\subsection{网络生活中的道德要求}
\subsection{大学生如何树立正确的择业观和创业观}
\subsection{怎样积极投身从德向善的道德实践}

\section{尊法、学法、守法、用法}
\subsection{法律的含义与特征}
\subsection{我国社会主义法律的本质特征}
\subsection{宪法的地位、基本原则和制度}
\subsection{全面依法治国的十六字方针}
\subsection{走中国特色社会主义法治道路的“五个坚持”}
\subsection{法治思维的内容、基本内涵及其培养}
